\chapter{Construção do Veículo F1TENTH}

Este capítulo detalha o processo de desenvolvimento e construção do veículo
para a competição F1TENTH, abrangendo desde a seleção de componentes até a
integração dos sistemas e validação do protótipo. Cada decisão de projeto foi
tomada considerando tanto as regulamentações da competição quanto os objetivos
de desempenho estabelecidos.

O objetivo principal foi criar um veículo que não apenas atendesse às
especificações técnicas obrigatórias, mas que também oferecesse uma plataforma
flexível para o desenvolvimento e teste de algoritmos avançados de percepção,
planejamento e controle.

\section{Seleção de Componentes}

A seleção de componentes foi realizada considerando múltiplos fatores como
conformidade com as regulamentações, desempenho, disponibilidade, custo e
potencial de integração. Cada componente foi avaliado quanto à sua contribuição
para os objetivos gerais do projeto e complementaridade com outros elementos do
sistema.

\subsection{Chassi e Sistema Mecânico}

Para a estrutura base do veículo, foi selecionado o chassi modelo D5 da marca
Emida, específico para a escala 1:10. A escolha deste chassi foi baseada em
suas características e compatibilidade com os requisitos da competição F1TENTH.

As especificações técnicas principais do chassi D5 incluem:
\begin{itemize}
      \item Dimensões: Comprimento de 38.2 cm, largura de 19.5 cm e distância entre eixos
            de 257 mm.
      \item Peso: Aproximadamente 0.78 kg (sem eletrônica), proporcionando uma base leve.
      \item Materiais: Estrutura principal em fibra de carbono, com componentes mecânicos
            em metal, oferecendo um bom equilíbrio entre rigidez e peso.
      \item Configuração: Posição do motor central (mid-motor), favorecendo a distribuição
            de peso.
      \item Suspensão: Sistema independente na dianteira e duplo braço (double wishbone) na
            traseira, com amortecedores ajustáveis, permitindo adaptação a diferentes
            condições de pista.
      \item Compatibilidade: Projetado para superfícies on-road e compatível com
            componentes eletrônicos padrão da escala 1:10.
      \item Ajustabilidade: Múltiplas posições para montagem da bateria e do motor,
            permitindo otimizar o centro de gravidade.
\end{itemize}

A estrutura em fibra de carbono garante a rigidez necessária para suportar as
cargas dinâmicas durante a operação em alta velocidade, enquanto a configuração
mid-motor e as opções de ajuste contribuem para a otimização do balanceamento e
da dirigibilidade do veículo.

\subsection{Sistema de Propulsão}

Para o sistema de propulsão, foi selecionado o motor EZRUN 3665SD-3200KV-G3, um
motor brushless DC que atende às especificações da competição, oferecendo
excelente relação entre desempenho e conformidade regulatória.

\subsubsection{Motor EZRUN 3665SD-3200KV-G3}

O motor EZRUN 3665SD-3200KV-G3 foi escolhido por suas características técnicas
alinhadas com os requisitos da competição:

\begin{itemize}
      \item Rating KV de 3200KV, dentro do limite regulamentar equivalente ao Velineon
            3500KV
      \item Compatibilidade com baterias 2-3S LiPo, comum em veículos RC em escala 1:10
      \item Dimensões físicas (37mm de diâmetro e 65.8mm de comprimento) adequadas para
            instalação no chassi padrão
      \item Construção robusta com carcaça de alumínio usinado em CNC e rolamentos de alta
            precisão
      \item Sistema de refrigeração otimizado para operação contínua em alta carga
      \item Classificação IP-67, oferecendo proteção contra água e poeira, importante para
            a durabilidade durante testes e competições
\end{itemize}

O motor apresenta um equilíbrio ideal entre torque e velocidade, fundamental
para o desempenho tanto em acelerações rápidas quanto em trechos de velocidade
constante nos circuitos da competição. A compatibilidade com sensores hall
integrados também facilita o controle preciso de velocidade necessário para
algoritmos de navegação autônoma.

\subsection{Sistema de Controle Eletrônico}

Para o controle eletrônico do motor, foi escolhido o Makerbase VESC MINI 6.7,
um controlador avançado que permite implementação de estratégias sofisticadas
de controle do motor.

\subsubsection{Makerbase VESC MINI 6.7}

O VESC MINI 6.7 foi selecionado pelas seguintes características:

\begin{itemize}
      \item Dimensões compactas (46.5mm x 36.5mm x 12mm) e peso reduzido (aproximadamente
            30g), ideais para aplicações em veículos em escala 1:10
      \item Ampla faixa de tensão operacional (8V-60V DC), compatível com diversos sistemas
            de bateria
      \item Capacidade de corrente contínua de 40A, suficiente para o motor selecionado
      \item Microcontrolador STM32F405RGT6, oferecendo poder de processamento para
            implementação de algoritmos avançados de controle
      \item Suporte para operação com e sem sensores, permitindo flexibilidade no modo de
            controle
      \item Capacidade de frenagem regenerativa, importante para manobras precisas
      \item Múltiplas interfaces de comunicação (USB, CAN, UART, PPM, I2C), facilitando a
            integração com a unidade de processamento principal
      \item Proteções integradas contra sobretensão, subtensão, sobrecorrente e
            sobretemperatura
\end{itemize}

A escolha deste controlador permite a implementação de controle vetorial de
campo orientado (FOC), resultando em operação mais eficiente e precisa do
motor. A capacidade de programação através do VESC Tool possibilita ajustes
finos nos parâmetros de controle, adaptando o comportamento do veículo para
diferentes seções da pista e estratégias de corrida.

\subsection{Sistema de Sensoriamento}

O principal sensor para percepção do ambiente é o LiDAR YDLIDAR X4PRO, que
fornece dados de distância em 360 graus para navegação e detecção de
obstáculos.

\subsubsection{YDLIDAR X4PRO}

O LiDAR YDLIDAR X4PRO foi escolhido por suas características técnicas que
atendem aos requisitos da competição:

\begin{itemize}
      \item Varredura omnidirecional de 360°, permitindo detecção completa do ambiente ao
            redor do veículo
      \item Alcance de medição de 0.12-10m, adequado para ambientes de competição indoor
      \item Frequência de amostragem de até 5000Hz, fornecendo dados detalhados para
            algoritmos de percepção
      \item Frequência de varredura ajustável entre 6-12Hz, permitindo equilíbrio entre
            resolução e tempo de resposta
      \item Resolução angular entre 0.43° e 0.86° (dependendo da frequência), suficiente
            para detecção precisa dos limites da pista e outros veículos
      \item Alta resistência a interferências luminosas (até 40000 Lux), importante para
            ambientes com iluminação variável
      \item Classificação de segurança laser Class I, atendendo aos requisitos de segurança
            da competição
      \item Dimensões compactas e peso de 178g, facilitando a integração no veículo
\end{itemize}

Este sensor fornece dados cruciais para algoritmos de mapeamento, localização,
planejamento de trajetória e detecção de outros veículos durante as corridas
head-to-head. Sua capacidade de detecção omnidirecional é particularmente
importante para a implementação de estratégias defensivas e ofensivas durante
ultrapassagens.

\subsection{Unidade de Processamento}

Para processamento dos algoritmos de percepção, planejamento e controle, foi
selecionada a Raspberry Pi 4B como unidade computacional principal.

\subsubsection{Raspberry Pi 4B}

A Raspberry Pi 4B foi escolhida como plataforma de processamento principal
pelas seguintes razões:

\begin{itemize}
      \item Processador quad-core Cortex-A72 (ARM v8) 64-bit operando a 1.5GHz, oferecendo
            poder computacional suficiente para algoritmos de tempo real
      \item Disponibilidade de 4GB ou 8GB de RAM, dependendo da configuração, permitindo
            processamento de dados sensoriais complexos
      \item Múltiplas interfaces de comunicação (USB 3.0, Ethernet Gigabit, Wi-Fi,
            Bluetooth, GPIO), facilitando a integração com diversos sensores e atuadores
      \item Dimensões compactas e peso reduzido, adequados para instalação em veículos em
            escala 1:10
      \item Amplo suporte para ROS2 (Robot Operating System 2), framework utilizado para
            implementação dos algoritmos de navegação autônoma
      \item Comunidade ativa e extensa documentação, facilitando o desenvolvimento
      \item Custo acessível em comparação com outras plataformas de desempenho similar
\end{itemize}

A Raspberry Pi 4B executa o sistema operacional Ubuntu 22.04 LTS com ROS2
Humble Hawksbill, fornecendo um ambiente robusto para desenvolvimento e
execução dos algoritmos de controle. A integração com o VESC e o LiDAR é
realizada através de interfaces seriais e pacotes ROS2 específicos, criando uma
arquitetura de software modular e extensível.

\subsection{Sistema de Alimentação}

O sistema de alimentação do veículo foi projetado com duas fontes de energia
distintas para garantir a estabilidade operacional: uma bateria de alta
descarga para o sistema de propulsão e um power bank para a eletrônica de
controle e processamento.

\subsubsection{Bateria de Propulsão: HRB LiPo 3S 5000mAh 50C}

A alimentação principal do motor EZRUN e do controlador VESC é fornecida por
uma bateria HRB LiPo 3S de 5000mAh. Suas características são:
\begin{itemize}
      \item Tipo: Polímero de Lítio (LiPo)
      \item Configuração: 3 células em série (3S)
      \item Tensão Nominal: 11.1V
      \item Capacidade: 5000mAh
      \item Taxa de Descarga: 50C contínua (250A), 100C pico
      \item Dimensões: 155mm x 48mm x 24mm
      \item Peso: 376g
\end{itemize}
Esta bateria foi escolhida por sua alta capacidade e taxa de descarga, capaz de
fornecer a corrente necessária para o motor durante acelerações intensas,
garantindo o desempenho requerido na competição.

\subsubsection{Alimentação da Eletrônica: Essager Power Bank 15000mAh 65W}

Para alimentar a Raspberry Pi 4B, o LiDAR, servo motor e outros periféricos de
baixa tensão, foi utilizado um power bank Essager modelo Xuneng de 15000mAh.
Suas características relevantes incluem:
\begin{itemize}
      \item Capacidade: 15000mAh
      \item Potência Máxima de Saída: 65W
      \item Portas: 2x USB-C, 1x USB-A
      \item Peso: Aproximadamente 300g
      \item Funcionalidade Adicional: Display LED e cabo USB-C integrado.
\end{itemize}
O uso de um power bank dedicado para a eletrônica desacopla a alimentação da
Raspberry Pi das flutuações de tensão causadas pelo motor, garantindo uma
operação mais estável e prevenindo reinicializações inesperadas da unidade de
processamento. A capacidade de 15000mAh oferece autonomia suficiente para
sessões prolongadas de teste e operação.

\section{Projeto e Fabricação de Peças Customizadas}

Para acomodar os diversos componentes e garantir a integridade estrutural do
veículo, várias peças customizadas foram projetadas e fabricadas utilizando
tecnologia de impressão 3D.

\subsection{Suportes e Estruturas Principais}

[Esta seção será completada posteriormente com detalhes sobre o suporte para a unidade de processamento e controlador eletrônico.]

\subsection{Sistema de Montagem e Integração}

[Esta seção será completada posteriormente com informações sobre como as peças customizadas se integram ao chassi e entre si, incluindo considerações sobre manutenção e acesso aos componentes.]

\subsection{Processo de Fabricação}

[Esta seção será completada posteriormente com detalhes sobre os parâmetros de impressão 3D, materiais utilizados e técnicas de pós-processamento.]

\section{Arquitetura de Hardware do Veículo}

 [Esta seção será completada posteriormente com um diagrama detalhado da arquitetura de hardware e explicações sobre as conexões entre componentes.]

\section{Arquitetura de Software do Veículo}

A arquitetura de software do veículo é baseada em ROS2, implementando uma
estrutura modular que facilita o desenvolvimento, teste e otimização dos
algoritmos de controle.

\subsection{Integração do VESC com ROS2}

A integração do controlador VESC MINI 6.7 com o sistema ROS2 é realizada
utilizando os pacotes ROS2 mantidos pela comunidade F1TENTH, especificamente o
metapacote \texttt{vesc}\footnote{Repositório F1TENTH VESC ROS2 Driver:
      \url{https://github.com/f1tenth/vesc/tree/ros2}}. Este metapacote inclui:
\begin{itemize}
      \item \texttt{vesc\_driver}: O nó principal que gerencia a comunicação serial com o
            VESC via USB (tipicamente na porta \texttt{/dev/ttyACM0}).
      \item \texttt{vesc\_msgs}: Define os tipos de mensagens customizadas para
            comunicação com o VESC, como \texttt{VescStateStamped} para telemetria e
            comandos.
      \item \texttt{vesc\_ackermann}: Converte comandos de \texttt{AckermannDriveStamped} (velocidade
            linear e ângulo de direção) em comandos de baixo nível compreensíveis pelo
            VESC (velocidade do motor em ERPM e posição do servo).
\end{itemize}

A comunicação se dá da seguinte forma:
\begin{itemize}
      \item Publicação de Estado: O nó \texttt{vesc\_driver} lê dados do VESC via serial e
            publica a telemetria (tensão, corrente, ERPM, temperatura, etc.) no tópico
            \texttt{/sensors/core} (tipo \texttt{vesc\_msgs/VescStateStamped}) e a posição
            do servo (se configurado) em \texttt{/sensors/servo\_position\_command}.
      \item Recebimento de Comandos: O nó \texttt{vesc\_ackermann} recebe comandos de alto
            nível de velocidade e direção no tópico \texttt{/drive} (tipo
            \texttt{ackermann\_msgs/AckermannDriveStamped}).
      \item Conversão e Envio: \texttt{vesc\_ackermann} converte os comandos de Ackermann
            em comandos de ERPM para o motor e posição para o servo, publicando-os nos
            tópicos \texttt{/commands/motor/speed} e \texttt{/commands/servo/position}
            respectivamente.
      \item Atuação: O nó \texttt{vesc\_driver} subscreve a esses tópicos de comando e
            envia as instruções correspondentes ao VESC via comunicação serial.
\end{itemize}

A configuração do sistema, incluindo a porta serial, ganhos de conversão
(velocidade para ERPM, ângulo para posição do servo) e limites operacionais, é
gerenciada através de arquivos de parâmetros YAML (como o
\texttt{vesc\_config.yaml} no pacote \texttt{vesc\_driver}), carregados durante
a inicialização dos nós com \texttt{ros2 launch}.

Essa arquitetura modular permite que os algoritmos de planejamento e controle
operem em um nível de abstração mais alto, enviando comandos
\texttt{AckermannDriveStamped}, enquanto os pacotes \texttt{vesc} cuidam da
comunicação de baixo nível e da conversão necessária para o hardware
específico.

\subsection{Integração do LiDAR com ROS2}

A integração do sensor LiDAR YDLIDAR X4PRO com o sistema ROS2 é realizada
através do pacote oficial \texttt{ydlidar\_ros2\_driver}\footnote{Repositório
      oficial do driver: \url{https://github.com/YDLIDAR/ydlidar_ros2_driver}}. Este
driver atua como uma ponte entre o hardware do sensor e o framework ROS2,
permitindo que os dados de varredura laser sejam processados e utilizados pelos
algoritmos de navegação.

Principais aspectos da integração:
\begin{itemize}
      \item Comunicação: O driver estabelece comunicação com o LiDAR através da porta
            serial (tipicamente \texttt{/dev/ttyUSB0} ou um alias como
            \texttt{/dev/ydlidar}) na Raspberry Pi. Os parâmetros de comunicação, como
            \texttt{baudrate} (230400 para o X4PRO) e \texttt{port}, são configuráveis.
      \item Publicação de Dados: O driver processa os dados brutos do LiDAR e os publica no
            tópico ROS2 \texttt{/scan}. A mensagem publicada é do tipo
            \texttt{sensor\_msgs/LaserScan}, um formato padrão em ROS para dados de
            varredura laser 2D. Esta mensagem contém informações como ângulos, distâncias e
            intensidades (se aplicável) para cada ponto da varredura.
      \item Configuração: Parâmetros específicos do LiDAR, como faixa angular
            (\texttt{angle\_min}, \texttt{angle\_max}), limites de alcance
            (\texttt{range\_min}, \texttt{range\_max}), frequência de varredura
            (\texttt{frequency}), e \texttt{frame\_id} (identificador da coordenada TF,
            e.g., \texttt{laser\_frame}), são definidos em arquivos de configuração (YAML)
            ou diretamente em arquivos de inicialização (arquivos \texttt{.launch.py}).
      \item Execução: O nó do driver é iniciado utilizando \texttt{ros2 launch},
            tipicamente com um arquivo \texttt{launch.py} customizado que carrega os
            parâmetros corretos para o X4PRO (e.g., \texttt{ydlidar\_launch.py}
            referenciado no repositório).
      \item Serviços: O driver também oferece serviços ROS2 básicos, como
            \texttt{start\_scan} e \texttt{stop\_scan}, permitindo ligar e desligar o
            sensor programaticamente.
\end{itemize}

Essa integração fornece os dados de percepção essenciais, no formato
\texttt{sensor\_msgs/LaserScan}, para os nós ROS2 responsáveis pelo mapeamento
SLAM (Simultaneous Localization and Mapping), localização (como AMCL --
Adaptive Monte Carlo Localization), planejamento de trajetória e detecção de
obstáculos. A padronização da mensagem facilita a interoperabilidade com outros
pacotes ROS2 do ecossistema de navegação.

\subsection{Algoritmos de Percepção e Controle}

[Esta seção será completada posteriormente com informações sobre os algoritmos implementados para mapeamento, localização, planejamento de trajetória e controle do veículo.]
