\chapter{Desenvolvimento e Construção do Veículo F1TENTH}

Este capítulo detalha o processo de desenvolvimento e construção do veículo
para a competição F1TENTH, abrangendo desde a seleção de componentes até a
integração dos sistemas e validação do protótipo. Cada decisão de projeto foi
tomada considerando tanto as regulamentações da competição quanto os objetivos
de desempenho estabelecidos.

\section{Visão Geral do Projeto}

O desenvolvimento do veículo F1TENTH seguiu uma abordagem sistemática, dividida
em quatro fases principais:

\begin{enumerate}
    \item Análise das regulamentações e requisitos da competição
    \item Seleção de componentes e arquitetura do sistema
    \item Projeto e fabricação de peças customizadas
    \item Integração de hardware e software e testes de validação
\end{enumerate}

O objetivo principal foi criar um veículo que não apenas atendesse às
especificações técnicas obrigatórias, mas que também oferecesse uma plataforma
flexível para o desenvolvimento e teste de algoritmos avançados de percepção,
planejamento e controle.

\section{Seleção de Componentes}

A seleção de componentes foi realizada considerando múltiplos fatores como
conformidade com as regulamentações, desempenho, disponibilidade, custo e
potencial de integração. Cada componente foi avaliado quanto à sua contribuição
para os objetivos gerais do projeto e complementaridade com outros elementos do
sistema.

\subsection{Chassi e Sistema Mecânico}

[Esta seção será completada posteriormente com informações sobre o chassi selecionado, modificações realizadas e justificativas para as escolhas.]

\subsection{Sistema de Propulsão}

Para o sistema de propulsão, foi selecionado o motor EZRUN 3665SD-3200KV-G3, um
motor brushless DC que atende às especificações da competição, oferecendo
excelente relação entre desempenho e conformidade regulatória.

\subsubsection{Motor EZRUN 3665SD-3200KV-G3}

O motor EZRUN 3665SD-3200KV-G3 foi escolhido por suas características técnicas
alinhadas com os requisitos da competição:

\begin{itemize}
    \item Rating KV de 3200KV, dentro do limite regulamentar equivalente ao Velineon
          3500KV
    \item Compatibilidade com baterias 2-3S LiPo, comum em veículos RC em escala 1:10
    \item Dimensões físicas (37mm de diâmetro e 65.8mm de comprimento) adequadas para
          instalação no chassi padrão
    \item Construção robusta com carcaça de alumínio usinado em CNC e rolamentos de alta
          precisão
    \item Sistema de refrigeração otimizado para operação contínua em alta carga
    \item Classificação IP-67, oferecendo proteção contra água e poeira, importante para
          a durabilidade durante testes e competições
\end{itemize}

O motor apresenta um equilíbrio ideal entre torque e velocidade, fundamental
para o desempenho tanto em acelerações rápidas quanto em trechos de velocidade
constante nos circuitos da competição. A compatibilidade com sensores hall
integrados também facilita o controle preciso de velocidade necessário para
algoritmos de navegação autônoma.

\subsection{Sistema de Controle Eletrônico}

Para o controle eletrônico do motor, foi escolhido o Makerbase VESC MINI 6.7,
um controlador avançado que permite implementação de estratégias sofisticadas
de controle do motor.

\subsubsection{Makerbase VESC MINI 6.7}

O VESC MINI 6.7 foi selecionado pelas seguintes características:

\begin{itemize}
    \item Dimensões compactas (46.5mm x 36.5mm x 12mm) e peso reduzido (aproximadamente
          30g), ideais para aplicações em veículos em escala 1:10
    \item Ampla faixa de tensão operacional (8V-60V DC), compatível com diversos sistemas
          de bateria
    \item Capacidade de corrente contínua de 40A, suficiente para o motor selecionado
    \item Microcontrolador STM32F405RGT6, oferecendo poder de processamento para
          implementação de algoritmos avançados de controle
    \item Suporte para operação com e sem sensores, permitindo flexibilidade no modo de
          controle
    \item Capacidade de frenagem regenerativa, importante para manobras precisas
    \item Múltiplas interfaces de comunicação (USB, CAN, UART, PPM, I2C), facilitando a
          integração com a unidade de processamento principal
    \item Proteções integradas contra sobretensão, subtensão, sobrecorrente e
          sobretemperatura
\end{itemize}

A escolha deste controlador permite a implementação de controle vetorial de
campo orientado (FOC), resultando em operação mais eficiente e precisa do
motor. A capacidade de programação através do VESC Tool possibilita ajustes
finos nos parâmetros de controle, adaptando o comportamento do veículo para
diferentes seções da pista e estratégias de corrida.

\subsection{Sistema de Sensoriamento}

O principal sensor para percepção do ambiente é o LiDAR YDLIDAR X4PRO, que
fornece dados de distância em 360 graus para navegação e detecção de
obstáculos.

\subsubsection{YDLIDAR X4PRO}

O LiDAR YDLIDAR X4PRO foi escolhido por suas características técnicas que
atendem aos requisitos da competição:

\begin{itemize}
    \item Varredura omnidirecional de 360°, permitindo detecção completa do ambiente ao
          redor do veículo
    \item Alcance de medição de 0.12-10m, adequado para ambientes de competição indoor
    \item Frequência de amostragem de até 5000Hz, fornecendo dados detalhados para
          algoritmos de percepção
    \item Frequência de varredura ajustável entre 6-12Hz, permitindo equilíbrio entre
          resolução e tempo de resposta
    \item Resolução angular entre 0.43° e 0.86° (dependendo da frequência), suficiente
          para detecção precisa dos limites da pista e outros veículos
    \item Alta resistência a interferências luminosas (até 40000 Lux), importante para
          ambientes com iluminação variável
    \item Classificação de segurança laser Class I, atendendo aos requisitos de segurança
          da competição
    \item Dimensões compactas e peso de 178g, facilitando a integração no veículo
\end{itemize}

Este sensor fornece dados cruciais para algoritmos de mapeamento, localização,
planejamento de trajetória e detecção de outros veículos durante as corridas
head-to-head. Sua capacidade de detecção omnidirecional é particularmente
importante para a implementação de estratégias defensivas e ofensivas durante
ultrapassagens.

\subsection{Unidade de Processamento}

Para processamento dos algoritmos de percepção, planejamento e controle, foi
selecionada a Raspberry Pi 4B como unidade computacional principal.

\subsubsection{Raspberry Pi 4B}

A Raspberry Pi 4B foi escolhida como plataforma de processamento principal
pelas seguintes razões:

\begin{itemize}
    \item Processador quad-core Cortex-A72 (ARM v8) 64-bit operando a 1.5GHz, oferecendo
          poder computacional suficiente para algoritmos de tempo real
    \item Disponibilidade de 4GB ou 8GB de RAM, dependendo da configuração, permitindo
          processamento de dados sensoriais complexos
    \item Múltiplas interfaces de comunicação (USB 3.0, Ethernet Gigabit, Wi-Fi,
          Bluetooth, GPIO), facilitando a integração com diversos sensores e atuadores
    \item Dimensões compactas e peso reduzido, adequados para instalação em veículos em
          escala 1:10
    \item Amplo suporte para ROS2 (Robot Operating System 2), framework utilizado para
          implementação dos algoritmos de navegação autônoma
    \item Comunidade ativa e extensa documentação, facilitando o desenvolvimento
    \item Custo acessível em comparação com outras plataformas de desempenho similar
\end{itemize}

A Raspberry Pi 4B executa o sistema operacional Ubuntu 22.04 LTS com ROS2
Humble Hawksbill, fornecendo um ambiente robusto para desenvolvimento e
execução dos algoritmos de controle. A integração com o VESC e o LiDAR é
realizada através de interfaces seriais e pacotes ROS2 específicos, criando uma
arquitetura de software modular e extensível.

\subsection{Sistema de Alimentação}

[Esta seção será completada posteriormente com informações sobre a bateria utilizada, sistema de distribuição de energia e considerações sobre autonomia.]

\subsection{Sensores Adicionais}

[Esta seção será completada posteriormente com informações sobre sensores complementares como IMU, encoders, câmeras ou quaisquer outros sensores utilizados.]

\section{Projeto e Fabricação de Peças Customizadas}

Para acomodar os diversos componentes e garantir a integridade estrutural do
veículo, várias peças customizadas foram projetadas e fabricadas utilizando
tecnologia de impressão 3D.

\subsection{Metodologia de Projeto}

[Esta seção será completada posteriormente com informações sobre o processo de design, software utilizado, considerações de projeto e metodologia de validação.]

\subsection{Suportes e Estruturas Principais}

\subsubsection{Base para Raspberry Pi e VESC}

[Esta seção será completada posteriormente com detalhes sobre o suporte para a unidade de processamento e controlador eletrônico.]

\subsubsection{Suporte para LiDAR}

[Esta seção será completada posteriormente com informações sobre o suporte do sensor LiDAR, incluindo considerações sobre posicionamento e calibração.]

\subsubsection{Para-choques e Estruturas de Proteção}

[Esta seção será completada posteriormente com detalhes sobre os para-choques frontais e traseiros conforme exigido pelas regulamentações.]

\subsection{Sistema de Montagem e Integração}

[Esta seção será completada posteriormente com informações sobre como as peças customizadas se integram ao chassi e entre si, incluindo considerações sobre manutenção e acesso aos componentes.]

\subsection{Processo de Fabricação}

[Esta seção será completada posteriormente com detalhes sobre os parâmetros de impressão 3D, materiais utilizados e técnicas de pós-processamento.]

\section{Integração de Sistemas}

Esta seção descreve como os diversos componentes foram integrados para formar
um sistema coeso, abordando tanto aspectos de hardware quanto de software.

\subsection{Arquitetura de Hardware}

[Esta seção será completada posteriormente com um diagrama detalhado da arquitetura de hardware e explicações sobre as conexões entre componentes.]

\subsection{Arquitetura de Software}

A arquitetura de software do veículo é baseada em ROS2, implementando uma
estrutura modular que facilita o desenvolvimento, teste e otimização dos
algoritmos de controle.

\subsubsection{Integração do VESC com ROS2}

Para integrar o controlador VESC MINI 6.7 com o sistema ROS2, foi desenvolvido
um driver específico baseado em comunicação serial. Este driver implementa:

\begin{itemize}
    \item Comunicação bidirecional entre a Raspberry Pi e o VESC através de porta serial
          (UART)
    \item Publicação de tópicos ROS2 para estados do motor (velocidade, corrente,
          temperatura)
    \item Subscrição a tópicos de comando para controle de velocidade ou torque
    \item Conversão entre unidades físicas (m/s) e unidades do motor (ERPM)
    \item Configuração de parâmetros dinâmicos através de serviços ROS2
    \item Mecanismos de diagnóstico e recuperação de erros
\end{itemize}

O código implementa uma classe `VescInterface` que gerencia a comunicação e os
tópicos ROS2 relacionados, permitindo controle preciso do motor baseado em
comandos de alto nível provenientes dos algoritmos de navegação.

\subsubsection{Integração do LiDAR com ROS2}

[Esta seção será completada posteriormente com detalhes sobre a integração do LiDAR YDLIDAR X4PRO com ROS2, incluindo pacotes utilizados e configurações específicas.]

\subsubsection{Algoritmos de Percepção e Controle}

[Esta seção será completada posteriormente com informações sobre os algoritmos implementados para mapeamento, localização, planejamento de trajetória e controle do veículo.]

\section{Testes e Validação}

 [Esta seção será completada posteriormente com metodologias de teste, resultados preliminares e processos de validação do veículo.]