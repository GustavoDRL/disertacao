% ----------------------------------------------------------
% Introdução 
% Capítulo sem numeração, mas presente no Sumário
% ----------------------------------------------------------

\chapter*[Introdução]{Introdução}
\addcontentsline{toc}{chapter}{Introdução}

O desenvolvimento de veículos autônomos representa uma das fronteiras mais
promissoras da engenharia moderna, combinando avanços em robótica, inteligência
artificial, sistemas de controle e processamento de sinais
\cite{Karaman2011Optimal}. Neste contexto, a categoria F1TENTH emerge como uma
plataforma educacional valiosa, permitindo que pesquisadores trabalhem com os
mesmos princípios e desafios encontrados em veículos autônomos em escala real,
porém em um ambiente mais controlado e acessível \cite{OKelly2020F1TENTH}.

A relevância do desenvolvimento de veículos autônomos de corrida em escala
reduzida se estabelece em múltiplas dimensões. Do ponto de vista técnico, os
desafios enfrentados no desenvolvimento de um veículo F1TENTH espelham, em
menor escala, os problemas encontrados na indústria automotiva atual. Segundo
\cite{Babu2020Simulator}, a necessidade de desenvolver sistemas robustos de
percepção, planejamento de trajetória e controle em tempo real reflete
diretamente os desafios enfrentados por empresas que trabalham com veículos
autônomos em escala real.

A escolha da plataforma F1TENTH é particularmente apropriada por seu caráter
open-source e pela existência de uma comunidade internacional ativa. Como
destacam \cite{Agnihotri2020Teaching}, esta característica não apenas facilita
o acesso a recursos e conhecimentos, mas também permite que os resultados
obtidos possam ser compartilhados e utilizados por outros pesquisadores e
estudantes, multiplicando o impacto do trabalho desenvolvido. Do ponto de vista
prático, o desenvolvimento de um veículo autônomo em escala reduzida apresenta
vantagens significativas em termos de custos, segurança e facilidade de
experimentação. \cite{Wang2020LMPC} demonstram que a plataforma permite a
realização de testes e validações de forma mais ágil e segura do que seria
possível com veículos em escala real, possibilitando uma iteração mais rápida
no desenvolvimento e aperfeiçoamento dos algoritmos e sistemas.

Além disso, o projeto se alinha com as tendências atuais da indústria
automotiva, onde há uma crescente demanda por profissionais com conhecimento em
sistemas autônomos. \cite{Stachowicz2023FastRLAP} ressaltam que a experiência
adquirida no desenvolvimento deste tipo de trabalho pode contribuir
significativamente para a formação de profissionais melhor preparados para
atuar neste mercado em expansão.

Neste contexto, este trabalho tem como objetivo principal desenvolver um
veículo autônomo de corrida seguindo as especificações da categoria F1TENTH,
abordando aspectos desde a construção do modelo físico até a implementação dos
sistemas de percepção, planejamento de trajetória e controle. O desenvolvimento
será realizado considerando as questões de segurança e verificação formal
destacadas por \cite{Ivanov2020Verifying}, que enfatizam a importância da
validação rigorosa em sistemas autônomos.

O trabalho está organizado em capítulos que abordam: fundamentação teórica,
detalhando os conceitos básicos necessários; desenvolvimento do hardware,
descrevendo a plataforma mecânica e componentes eletrônicos; desenvolvimento do
software, apresentando a arquitetura e implementação dos algoritmos; e
resultados e discussão, analisando o desempenho do veículo e comparando com
trabalhos similares.

A estrutura apresentada permite uma abordagem sistemática do problema,
contribuindo para o avanço do conhecimento em uma área tecnológica estratégica,
enquanto promove a integração de diferentes áreas do conhecimento em
engenharia. O resultado esperado não é apenas um veículo autônomo funcional,
mas também um conjunto de conhecimentos e experiências que podem beneficiar
futuros projetos e pesquisas na área de sistemas autônomos.

\section*{Justificativa}\label{sec:Justificativa}
\addcontentsline{toc}{section}{Justificativa}

O desenvolvimento de veículos autônomos representa uma das fronteiras mais
promissoras da engenharia moderna, combinando avanços em robótica, inteligência
artificial, sistemas de controle e processamento de sinais. Neste contexto, a
categoria F1TENTH emerge como uma plataforma educacional valiosa, permitindo
que estudantes e pesquisadores trabalhem com os mesmos princípios e desafios
encontrados em veículos autônomos em escala real, porém em um ambiente mais
controlado e acessível.

A relevância deste trabalho se estabelece em múltiplas dimensões. Do ponto de
vista educacional, o desenvolvimento de um veículo autônomo de corrida oferece
uma oportunidade única de aplicar conhecimentos teóricos em um projeto prático
e desafiador. Os conceitos fundamentais de engenharia de controle,
processamento de sinais, programação em tempo real e robótica convergem em um
único projeto, proporcionando uma experiência de aprendizado integrada e
multidisciplinar.

No âmbito tecnológico, os desafios enfrentados no desenvolvimento de um veículo
F1TENTH espelham, em menor escala, os problemas encontrados na indústria
automotiva atual. A necessidade de desenvolver sistemas robustos de percepção,
planejamento de trajetória e controle em tempo real reflete diretamente os
desafios enfrentados por empresas que trabalham com veículos autônomos em
escala real.

Do ponto de vista prático, o desenvolvimento de um veículo autônomo em escala
reduzida apresenta vantagens significativas em termos de custos, segurança e
facilidade de experimentação. A plataforma permite a realização de testes e
validações de forma mais ágil e segura do que seria possível com veículos em
escala real, possibilitando uma iteração mais rápida no desenvolvimento e
aperfeiçoamento dos algoritmos e sistemas.

\section*{Motivação}\label{sec:motivacao}
\addcontentsline{toc}{section}{Motivação}

A escolha da plataforma F1TENTH é particularmente apropriada por seu caráter
open-source e pela existência de uma comunidade internacional ativa. Esta
característica não apenas facilita o acesso a recursos e conhecimentos, mas
também permite que os resultados obtidos possam ser compartilhados e utilizados
por outros pesquisadores e estudantes, multiplicando o impacto do trabalho
desenvolvido.

Além disso, o projeto se alinha com as tendências atuais da indústria
automotiva, onde há uma crescente demanda por profissionais com conhecimento em
sistemas autônomos. A experiência adquirida no desenvolvimento deste trabalho
pode contribuir significativamente para a formação de profissionais melhor
preparados para atuar neste mercado em expansão.

A competitividade inerente à categoria F1TENTH também adiciona um elemento
motivador importante ao projeto. A necessidade de otimizar o desempenho do
veículo para competições estimula a busca por soluções inovadoras e eficientes,
promovendo um ambiente de aprendizado dinâmico e desafiador.

Por fim, este trabalho se justifica pela sua contribuição para o avanço do
conhecimento em uma área tecnológica estratégica, oferecendo oportunidades de
desenvolvimento técnico e profissional, ao mesmo tempo em que promove a
integração de diferentes áreas do conhecimento em engenharia. O resultado
esperado não é apenas um veículo autônomo funcional, mas também um conjunto de
conhecimentos e experiências que podem beneficiar futuros projetos e pesquisas
na área de sistemas autônomos.

\section*{Objetivos}\label{sec:objetivos}
\addcontentsline{toc}{section}{Objetivos}

\lipsum[36]
