\chapter{Considerações finais}

A primeira parte deste trabalho de conclusão de curso explorou os fundamentos
teóricos, ferramentas e regulamentações que constituem o ecossistema da
competição F1TENTH. Através desta análise, evidencia-se que a plataforma
F1TENTH representa não apenas um ambiente competitivo, mas um laboratório
educacional que sintetiza os desafios centrais da mobilidade autônoma
contemporânea.

\begin{itemize}
    \item \textbf{Integração técnica e prática:} A implementação de sistemas ROS2 em plataformas com restrições computacionais reflete fielmente os compromissos necessários no desenvolvimento de veículos autônomos comerciais, onde otimização de recursos e desempenho determinístico são requisitos inegociáveis.

    \item \textbf{Complexidade progressiva:} A estrutura da competição, evoluindo de desafios de tempo individual para confrontos diretos, cria um percurso pedagógico que gradualmente introduz as complexidades da tomada de decisão em ambientes dinâmicos.

    \item \textbf{Robustez algorítmica:} O foco em equalizar vantagens mecânicas direciona o desenvolvimento para soluções algorítmicas sofisticadas, priorizando eficiência computacional e resiliência a condições adversas.

    \item \textbf{Fusão sensorial e percepção:} As limitações deliberadas nos sistemas de sensoriamento estimulam abordagens inovadoras para fusão de dados, filtragem de ruído e extração de características em tempo real.

    \item \textbf{Prototipagem acelerada:} A escala reduzida permite ciclos rápidos de desenvolvimento e experimentação que seriam proibitivamente custosos e arriscados em veículos de tamanho real.
\end{itemize}

As restrições impostas pelas regulamentações técnicas, longe de limitarem a
inovação, criam um ambiente controlado onde soluções algorítmicas criativas
ganham precedência sobre vantagens de hardware, espelhando os compromissos
enfrentados na indústria automotiva. Esta abordagem estabelece um microcosmo
ideal para a aplicação prática dos conhecimentos teóricos de robótica móvel,
sistemas embarcados e inteligência artificial em um contexto tangível e
mensurável.

A próxima etapa deste trabalho se concentrará na implementação prática destes
conceitos, aplicando o conhecimento teórico adquirido ao desenvolvimento de um
veículo F1TENTH funcional, com ênfase particular na navegação autônoma robusta
e na operação próxima aos limites dinâmicos do sistema.