\chapter{Materiais e Métodos}\label{cap:ferramentas}

\section{Fundamentação Teórica}

\subsection{Odometria: Estimativa de Pose por Sensoriamento de Movimento}

Em sistemas de robótica móvel, a capacidade de estimar a pose do robô (posição
e orientação) ao longo do tempo é fundamental para a navegação autônoma. Uma
das técnicas mais básicas e amplamente utilizadas para esta finalidade é a
odometria. A odometria, em sua essência, é um método que utiliza dados de
sensores de movimento do próprio robô para estimar a sua pose relativa a um
ponto de partida conhecido \cite{thrun2005}. Diferentemente de sistemas de
localização global (GPS), a odometria é um sistema de localização relativo, que
acumula erros ao longo do tempo, mas que fornece uma estimativa contínua da
pose do robô em alta frequência, sendo crucial para diversas tarefas de
controle e planejamento em robótica móvel.

\subsubsection{Princípios da Odometria de Rodas}

Para robôs móveis com rodas, a forma mais comum de odometria é a odometria de
rodas. Esta técnica baseia-se na medição do movimento rotacional das rodas do
robô, tipicamente através de encoders rotativos acoplados aos eixos das rodas.
Os encoders fornecem pulsos elétricos a cada incremento de rotação da roda,
permitindo quantificar o deslocamento angular de cada roda.

Com base no conhecimento da geometria do robô (distância entre as rodas, raio
das rodas) e nas leituras dos encoders, é possível estimar o movimento do robô.
Através de modelos cinemáticos simples, o movimento rotacional das rodas é
convertido em deslocamento linear e angular do robô. Por exemplo, em um robô
diferencial, a diferença na rotação das rodas direita e esquerda indica a
rotação do robô, enquanto a média da rotação indica o movimento linear para
frente ou para trás.

\subsubsection{Odometria e Sistemas de Coordenadas no ROS2}

No contexto do ROS2, a odometria desempenha um papel fundamental na manutenção
do sistema de coordenadas \texttt{odom}. Conforme mencionado anteriormente, o
frame \texttt{odom} é um sistema de coordenadas local que fornece estimativas
de posição relativa. A odometria é tipicamente o principal componente
responsável por fornecer a transformação entre o frame \texttt{odom} e o frame
\texttt{base\_link}, representando a pose do robô em relação ao referencial de
odometria.

É importante notar que o frame \texttt{odom} não é um sistema de coordenadas global e fixo como o frame \texttt{map}. O frame \texttt{odom} é um referencial local que se move e deriva ao longo do tempo, acompanhando a estimativa de pose da odometria.

\subsubsection{Limitações da Odometria}

Apesar de sua importância, a odometria é inerentemente propensa a erros. Como a
pose é estimada através da integração de pequenos movimentos incrementais, os
erros de medição acumulam-se ao longo do tempo, resultando no fenômeno
conhecido como deriva (\textit{drift}). As principais fontes de erro na
odometria de rodas incluem:

\begin{itemize}
    \item \textbf{Deslizamento das rodas (wheel slippage):} Ocorre quando as rodas patinam ou deslizam sem rotacionar de forma precisa sobre a superfície, levando a leituras incorretas dos encoders.
    \item \textbf{Irregularidades no terreno:} Superfícies irregulares ou obstáculos podem causar movimentos não intencionais das rodas, gerando erros na estimativa de movimento.
    \item \textbf{Imperfeições mecânicas e calibração:} Variações no diâmetro das rodas, desalinhamento dos encoders ou erros na calibração do sistema podem introduzir erros sistemáticos na odometria.
    \item \textbf{Resolução dos encoders:} Encoders com baixa resolução podem não capturar pequenos movimentos rotacionais, limitando a precisão da odometria.
\end{itemize}

Devido à acumulação de erros, a odometria não é adequada para localização
global precisa a longo prazo. A pose estimada pela odometria torna-se cada vez
mais incerta com o tempo e a distância percorrida.

\subsubsection{Importância da Odometria na Navegação Robótica}

Apesar de suas limitações, a odometria continua sendo um componente essencial
em sistemas de navegação robótica. Suas principais vantagens incluem:

\begin{itemize}
    \item \textbf{Alta frequência de atualização:} A odometria fornece estimativas de pose em alta taxa, crucial para sistemas de controle em tempo real e planejamento reativo.
    \item \textbf{Independência de sensores externos:} A odometria utiliza apenas sensores internos do robô, funcionando em ambientes sem infraestrutura externa (como GPS ou beacons).
    \item \textbf{Baixo custo computacional:} Os algoritmos de odometria são computacionalmente leves, permitindo sua execução em tempo real em plataformas embarcadas.
\end{itemize}

Em sistemas de navegação autônoma robustos, a odometria é frequentemente
utilizada em conjunto com outras técnicas de sensoriamento e localização, como
SLAM (Simultaneous Localization and Mapping) e localização baseada em sensores
externos (e.g., LiDAR, câmeras). A odometria fornece uma estimativa inicial e
contínua da pose, que pode ser refinada e corrigida por algoritmos de
localização mais precisos, como o AMCL (Adaptive Monte Carlo Localization)
mencionado anteriormente, que frequentemente utiliza a odometria como uma de
suas fontes de informação.

\subsubsection{Odometria no ROS2 e Navigation2}

No ecossistema ROS2 e na biblioteca Navigation2, a odometria é um dado de
entrada fundamental para diversos módulos. O Nav2 espera receber dados de
odometria através de tópicos ROS, tipicamente no formato de mensagens
\texttt{nav\_msgs/Odometry}. Estes dados são utilizados por algoritmos de
localização, como AMCL, para estimar a pose do robô no mapa. Além disso, a
odometria pode ser combinada com dados de outros sensores, como IMUs (Unidades
de Medida Inercial), através de pacotes como o \texttt{robot\_localization},
para obter uma estimativa de pose mais robusta e precisa através de técnicas de
fusão de sensores.

\subsection{Base\_link: Frame de Referência Central do Robô}

Em arquiteturas robóticas, especialmente no contexto do ROS2 e da biblioteca
Navigation2, o frame \texttt{base\_link} assume um papel central como o sistema
de coordenadas fundamental atrelado ao corpo do robô. Conforme mencionado na
seção sobre sistemas de coordenadas \cite{Babu2020Simulator}, o
\texttt{base\_link} é um frame rígido fixado ao robô, servindo como ponto de
referência para a localização de todos os outros componentes e sensores do
sistema robótico.

\subsubsection{Definição e Significado do Base\_link}

O \texttt{base\_link} é definido como um sistema de coordenadas local e fixo em
relação ao corpo principal do robô. Tipicamente, a origem do
\texttt{base\_link} é localizada no centro geométrico do robô, ou em um ponto
de referência significativo para o design do robô. A orientação do
\texttt{base\_link} também é definida em relação ao robô, com os eixos X, Y e Z
alinhados com as direções de movimento e orientação do robô (por exemplo, X
para frente, Y para a esquerda, Z para cima).

É crucial entender que o \texttt{base\_link} é um frame abstrato e virtual. Ele não corresponde necessariamente a um componente físico específico do robô. Em vez disso, ele representa um ponto de referência conceitual que facilita a organização e a transformação de dados sensoriais e de controle dentro do sistema robótico.

\subsubsection{Importância do Base\_link na Arquitetura Robótica}

O \texttt{base\_link} é essencial por diversas razões na arquitetura de um robô
móvel. Como ponto de referência unificado, ele serve como um sistema de
coordenadas comum para integrar dados de diferentes sensores (encoders, IMUs,
LiDARs, câmeras) e atuadores do robô (motores, direção). As transformações de
todos os sensores e atuadores são tipicamente expressas em relação ao
\texttt{base\_link}, facilitando a fusão de dados e o controle coordenado do
robô.

Na representação da pose do robô, a posição no mundo (em frames como
\texttt{map} ou \texttt{odom}) é geralmente expressa como uma transformação em
relação ao \texttt{base\_link}. Isso significa que, ao se referir à "pose do
robô", geralmente se está falando da pose do \texttt{base\_link} em relação a
outro frame de referência.

Os módulos de navegação, como o Navigation2 no ROS2, utilizam o
\texttt{base\_link} como a representação da pose do robô para planejamento de
trajetória, controle e localização. Os comandos de velocidade e as informações
de sensores são interpretados e gerados em relação ao \texttt{base\_link}.

Em ambientes de simulação e ferramentas de modelagem CAD, o \texttt{base\_link}
simplifica a representação do robô como uma entidade coesa, facilitando a
definição de suas propriedades físicas e cinemáticas.

\subsubsection{Relação do Base\_link com Map e Odom}

O \texttt{base\_link} se relaciona diretamente com outros frames de coordenadas
cruciais na navegação robótica, como \texttt{map} e \texttt{odom}. A relação
entre estes frames segue a hierarquia \texttt{map} → \texttt{odom} →
\texttt{base\_link}. A transformação \texttt{map} → \texttt{odom} descreve a
pose do frame \texttt{odom} em relação ao frame global \texttt{map}, sendo
tipicamente estimada por algoritmos de localização global. A transformação
\texttt{odom} → \texttt{base\_link} descreve a pose do \texttt{base\_link} em
relação ao frame \texttt{odom}, sendo primariamente estimada pela odometria.

Em essência, o \texttt{base\_link} é o frame mais próximo do robô físico,
enquanto o \texttt{odom} fornece uma estimativa de pose relativa baseada em
sensores de movimento, e o \texttt{map} representa o sistema de coordenadas
global e fixo do ambiente. A transformação entre \texttt{map} e \texttt{odom}
permite corrigir os erros acumulados na odometria, ancorando a pose do robô no
mapa global.

\subsubsection{Base\_footprint: Projeção no Plano do Chão}

É importante mencionar também o frame \texttt{base\_footprint}, que é a projeção do \texttt{base\_link} no plano do chão. O \texttt{base\_footprint} é frequentemente utilizado em planejadores de trajetória e controladores de movimento que operam em um plano 2D. A transformação entre \texttt{base\_link} e \texttt{base\_footprint} é tipicamente uma translação vertical ao longo do eixo Z, representando a altura do \texttt{base\_link} em relação ao solo.

\subsubsection{Considerações Práticas e Utilização no ROS2}

No ROS2, a configuração correta do \texttt{base\_link} e suas transformações é
fundamental para o funcionamento adequado da Navigation2 e de outros pacotes
robóticos. Tipicamente, o frame \texttt{base\_link} é definido no arquivo URDF
(Unified Robot Description Format) do robô, que descreve a estrutura física e
cinemática do robô. As transformações entre \texttt{base\_link} e outros frames
são gerenciadas pelo pacote tf2, que publica e mantém as relações entre os
sistemas de coordenadas em tempo real.

Ao desenvolver aplicações robóticas no ROS2, é essencial garantir que o
\texttt{base\_link} esteja corretamente definido e que as transformações para
outros frames, especialmente \texttt{odom} e \texttt{map}, sejam precisas e
consistentes. A configuração e calibração adequadas dos sensores de odometria e
dos algoritmos de localização são cruciais para obter uma estimativa de pose
robusta e confiável do \texttt{base\_link} no ambiente.

\subsection{Conceitos Básicos de Robótica Móvel e Navegação ROS2}

A navegação autônoma de robôs móveis representa um dos desafios fundamentais da
robótica moderna, exigindo a integração de múltiplos subsistemas para permitir
que um robô se mova de forma segura e eficiente em seu ambiente. No contexto do
ROS2 (Robot Operating System 2), a biblioteca Navigation2 (Nav2) fornece uma
arquitetura modular e extensível para implementação de navegação autônoma
\cite{OKelly2020F1TENTH}.

\subsubsection{Sistemas de Coordenadas e Transformações}

Um conceito fundamental para a navegação robótica é o sistema de coordenadas e
suas transformações. No ROS2, o pacote tf2 (Transform Frame 2) é responsável
por manter e gerenciar as relações entre diferentes sistemas de coordenadas do
robô. Segundo \cite{Babu2020Simulator}, os principais frames utilizados na
navegação incluem:

\begin{itemize}
    \item \texttt{map}: Sistema de coordenadas global fixo, representando o ambiente de navegação
    \item \texttt{odom}: Sistema de coordenadas local que fornece estimativas de posição relativa
    \item \texttt{base\_link}: Sistema de coordenadas fixo ao centro do robô
    \item \texttt{base\_footprint}: Projeção do \texttt{base\_link} no plano do chão
\end{itemize}

A relação entre estes frames é fundamental para a localização e navegação do
robô. O Nav2 utiliza estas transformações para converter medições de sensores e
comandos de controle entre diferentes sistemas de coordenadas, permitindo uma
navegação precisa e coerente.

\subsubsection{Mapeamento e Localização}

O processo de SLAM (Simultaneous Localization and Mapping) é essencial para a
navegação autônoma, permitindo que o robô construa um mapa do ambiente enquanto
simultaneamente se localiza nele. \cite{Wang2020LMPC} destacam que o Nav2
suporta diferentes algoritmos de SLAM, incluindo:

\begin{itemize}
    \item \textbf{Mapeamento}: Criação de representações do ambiente através de mapas de ocupação
    \item \textbf{Localização}: Estimativa da pose do robô utilizando algoritmos como AMCL (Adaptive Monte Carlo Localization)
    \item \textbf{Loop Closure}: Detecção e correção de inconsistências no mapa quando revisitando áreas conhecidas
\end{itemize}

\subsubsection{Planejamento de Trajetória}

O planejamento de trajetória é estruturado em duas camadas principais, conforme
descrito por \cite{Karaman2011Optimal}:

\begin{itemize}
    \item \textbf{Planejador Global}: Responsável por encontrar um caminho livre de obstáculos do ponto atual até o objetivo, utilizando algoritmos como:
          \begin{itemize}
              \item A* para busca em grafos
              \item RRT (Rapidly-exploring Random Trees) para espaços contínuos
              \item Navfn para navegação baseada em campos potenciais
          \end{itemize}

    \item \textbf{Planejador Local}: Responsável por gerar comandos de velocidade em tempo real, considerando:
          \begin{itemize}
              \item Desvio de obstáculos dinâmicos
              \item Restrições cinemáticas do robô
              \item Otimização local da trajetória
          \end{itemize}
\end{itemize}

\subsubsection{Controle e Execução}

O sistema de controle no Nav2 é responsável por converter os planos de
trajetória em comandos de movimento para os atuadores do robô.
\cite{Stachowicz2023FastRLAP} destacam os principais componentes:

\begin{itemize}
    \item \textbf{Controladores de Trajetória}: Implementam algoritmos como Pure Pursuit, DWA (Dynamic Window Approach) ou TEB (Timed Elastic Band)
    \item \textbf{Recovery Behaviors}: Estratégias para recuperação de situações de falha
    \item \textbf{Camada de Segurança}: Garante operação segura através de monitores de colisão e limites de velocidade
\end{itemize}

\subsubsection{Arquitetura de Navegação}

A arquitetura do Nav2 é construída sobre o conceito de Behavior Trees, que
segundo \cite{Ivanov2020Verifying}, oferece várias vantagens:

\begin{itemize}
    \item \textbf{Modularidade}: Facilita a adição e remoção de comportamentos
    \item \textbf{Reusabilidade}: Permite a composição de comportamentos complexos a partir de blocos básicos
    \item \textbf{Flexibilidade}: Suporta diferentes estratégias de navegação através da reconfiguração da árvore
\end{itemize}

A árvore de comportamentos coordena a execução dos diferentes módulos do
sistema, incluindo:
\begin{itemize}
    \item Planejamento de trajetória
    \item Controle de movimento
    \item Recuperação de falhas
    \item Monitoramento de estado
\end{itemize}

Esta arquitetura modular permite que desenvolvedores adaptem e estendam o
sistema para diferentes aplicações e plataformas robóticas, mantendo a robustez
e confiabilidade necessárias para navegação autônoma.

\subsection{Sistemas de Controle e Navegação Autônoma}

A navegação autônoma de veículos requer uma arquitetura de controle robusta que
integre diferentes níveis de abstração, desde o controle de baixo nível dos
atuadores até o planejamento estratégico de alto nível. De acordo com
\cite{Wang2020LMPC}, essa hierarquia de controle pode ser estruturada em três
camadas principais que trabalham em conjunto para garantir uma navegação segura
e eficiente. O campo das corridas autônomas, revisado por
\cite{Betz2022Survey}, apresenta desafios únicos nesta integração,
especialmente ao operar nos limites dinâmicos do veículo.

\subsubsection{Controle de Baixo Nível}

O controle de baixo nível é responsável pela execução direta dos comandos nos
atuadores do veículo. Para veículos com geometria Ackermann, como é o caso do
F1TENTH, \cite{Babu2020Simulator} descrevem que o sistema de controle precisa
gerenciar dois aspectos fundamentais:

\begin{itemize}
    \item \textbf{Controle de Direção}: Responsável por ajustar o ângulo das rodas dianteiras através do servomotor. O modelo cinemático da direção Ackermann pode ser expresso pela equação:
          \begin{equation}
              \tan(\delta) = \frac{L}{R}
          \end{equation}
          Onde $\delta$ é o ângulo de esterçamento, $L$ é a distância entre eixos e $R$ é o raio de curvatura instantâneo.

    \item \textbf{Controle de Velocidade}: Gerencia a velocidade longitudinal do veículo através do motor de tração. O modelo dinâmico simplificado pode ser representado por:
          \begin{equation}
              \dot{v} = \frac{F_t - F_r}{m}
          \end{equation}
          Onde $v$ é a velocidade, $F_t$ é a força de tração, $F_r$ representa as forças resistivas e $m$ é a massa do veículo.
\end{itemize}

\subsubsection{Controle Preditivo Baseado em Modelo (MPC)}

O controle preditivo baseado em modelo representa uma abordagem avançada para
navegação autônoma. \cite{Stachowicz2023FastRLAP} explicam que o MPC opera
resolvendo um problema de otimização em tempo real que considera:

\begin{itemize}
    \item \textbf{Modelo do Sistema}: Representa a dinâmica do veículo através de equações diferenciais:
          \begin{equation}
              \dot{x} = f(x,u)
          \end{equation}
          Onde $x$ é o estado do sistema e $u$ são as entradas de controle.

    \item \textbf{Função Objetivo}: Define os objetivos de controle, tipicamente incluindo:
          \begin{itemize}
              \item Minimização do erro de trajetória
              \item Suavidade dos comandos de controle
              \item Eficiência energética
          \end{itemize}

    \item \textbf{Restrições}: Limitações físicas e operacionais:
          \begin{itemize}
              \item Limites de velocidade e aceleração
              \item Restrições de esterçamento
              \item Limites de segurança
          \end{itemize}
\end{itemize}

\subsubsection{Controle Adaptativo e Aprendizado}

O controle adaptativo tem se mostrado particularmente relevante para aplicações
de corrida autônoma. \cite{Karaman2011Optimal} destacam que sistemas
adaptativos podem ajustar seus parâmetros em tempo real para:

\begin{itemize}
    \item \textbf{Compensar Incertezas}: Adaptação a variações nas condições da pista e dinâmica do veículo através de:
          \begin{itemize}
              \item Estimação online de parâmetros
              \item Ajuste automático de ganhos de controle
              \item Compensação de perturbações (por exemplo, \cite{Jain2020Bayesrace} exploram o
                    uso de Processos Gaussianos para corrigir modelos cinemáticos simples com base
                    na experiência prévia)
          \end{itemize}

    \item \textbf{Otimização de Desempenho}: Melhoria contínua através de:
          \begin{itemize}
              \item Aprendizado por reforço iterativo
              \item Ajuste baseado em experiência prévia
              \item Adaptação a diferentes condições de corrida
          \end{itemize}
\end{itemize}

\subsubsection{Arquitetura de Segurança}

\cite{Ivanov2020Verifying} enfatizam a importância de uma arquitetura de segurança robusta que inclua:

\begin{itemize}
    \item \textbf{Monitoramento de Estado}: Sistema que avalia continuamente:
          \begin{itemize}
              \item Integridade dos sensores
              \item Estado dos atuadores
              \item Consistência das medições
          \end{itemize}

    \item \textbf{Sistema de Supervisão}: Responsável por:
          \begin{itemize}
              \item Detecção de falhas
              \item Ativação de comportamentos de recuperação
              \item Garantia de operação dentro de limites seguros
          \end{itemize}

    \item \textbf{Camada de Verificação}: Implementa:
          \begin{itemize}
              \item Verificação formal de propriedades de segurança
              \item Validação de comandos de controle
              \item Garantias de estabilidade
          \end{itemize}
\end{itemize}

\subsubsection{Integração dos Sistemas}

A integração efetiva destes diferentes níveis de controle é crucial para o
desempenho global do sistema. \cite{OKelly2020F1TENTH} propõem uma arquitetura
que coordena:

\begin{itemize}
    \item \textbf{Comunicação entre Camadas}: Através de:
          \begin{itemize}
              \item Interfaces bem definidas
              \item Protocolos de mensagem padronizados
              \item Sincronização temporal precisa
          \end{itemize}

    \item \textbf{Gerenciamento de Estados}: Incluindo:
          \begin{itemize}
              \item Transições suaves entre modos de operação
              \item Tratamento de condições de exceção
              \item Coordenação de subsistemas
          \end{itemize}

    \item \textbf{Otimização Global}: Considerando:
          \begin{itemize}
              \item Objetivos de alto nível
              \item Restrições de segurança
              \item Eficiência operacional
          \end{itemize}
\end{itemize}

Esta estrutura hierárquica de controle permite que o veículo autônomo navegue
de forma segura e eficiente, adaptando-se a diferentes condições e cenários de
corrida enquanto mantém garantias de segurança e desempenho.

\subsection{Sensoriamento e Percepção do Ambiente}

O sistema de percepção em veículos autônomos de corrida desempenha um papel
fundamental na compreensão do ambiente e na tomada de decisões em tempo real.
Este sistema precisa processar informações de diversos sensores para criar uma
representação precisa e atualizada do ambiente, permitindo uma navegação segura
em altas velocidades.

\subsubsection{Sensores Principais}

No contexto da plataforma F1TENTH, \cite{OKelly2020F1TENTH} descrevem que os
principais sensores utilizados podem ser categorizados em três grupos
fundamentais:

\begin{itemize}
    \item O \textbf{LiDAR} (Light Detection and Ranging) representa o sensor primário
          para percepção do ambiente. Este sensor emite pulsos laser e mede o tempo de
          retorno do sinal refletido, criando uma nuvem de pontos bidimensional que
          representa a geometria do ambiente ao redor do veículo. A medição do LiDAR pode
          ser expressa matematicamente como:
          \begin{equation}
              d_i = \frac{c \cdot \Delta t_i}{2}
          \end{equation}
          Onde $d_i$ é a distância medida para o ponto $i$, $c$ é a velocidade da luz, e $\Delta t_i$ é o tempo de voo do pulso laser.

    \item A \textbf{Unidade de Medição Inercial (IMU)} fornece dados cruciais sobre o
          movimento do veículo. Segundo \cite{Stachowicz2023FastRLAP}, a IMU integra
          acelerômetros e giroscópios para medir:
          \begin{equation}
              \begin{bmatrix}
                  a_x \\
                  a_y \\
                  a_z
              \end{bmatrix} =
              \begin{bmatrix}
                  \ddot{x} \\
                  \ddot{y} \\
                  \ddot{z}
              \end{bmatrix} +
              \begin{bmatrix}
                  g_x \\
                  g_y \\
                  g_z
              \end{bmatrix}
          \end{equation}
          Onde $a$ representa as acelerações medidas, $\ddot{x}$ as acelerações reais do veículo, e $g$ a aceleração gravitacional.

    \item Os \textbf{encoders} nas rodas fornecem odometria, medindo o deslocamento do
          veículo através da rotação das rodas. A relação entre pulsos do encoder e
          deslocamento linear pode ser expressa como:
          \begin{equation}
              \Delta s = \frac{2\pi r}{N} \cdot n
          \end{equation}
          Onde $r$ é o raio da roda, $N$ é o número de pulsos por revolução, e $n$ é o número de pulsos contados.
\end{itemize}

\subsubsection{Fusão Sensorial}

A fusão sensorial é essencial para combinar dados de diferentes sensores e
criar uma estimativa mais precisa do estado do veículo e do ambiente.
\cite{Babu2020Simulator} explicam que o Filtro de Kalman Estendido (EKF) é
comumente utilizado para este propósito, operando em duas etapas:

\begin{itemize}
    \item \textbf{Predição}:
          \begin{align}
              \hat{x}_{k|k-1} & = f(\hat{x}_{k-1|k-1}, u_k)   \\
              P_{k|k-1}       & = F_k P_{k-1|k-1} F_k^T + Q_k
          \end{align}

    \item \textbf{Atualização}:
          \begin{align}
              K_k           & = P_{k|k-1} H_k^T (H_k P_{k|k-1} H_k^T + R_k)^{-1} \\
              \hat{x}_{k|k} & = \hat{x}_{k|k-1} + K_k(z_k - h(\hat{x}_{k|k-1}))
          \end{align}
\end{itemize}

\subsubsection{Processamento de Dados LiDAR}

O processamento dos dados do LiDAR é particularmente importante para a
navegação em alta velocidade. \cite{Wang2020LMPC} descrevem as principais
etapas deste processamento:

\begin{itemize}
    \item \textbf{Filtragem de Ruído}: Remoção de medições espúrias através de filtros estatísticos e temporais.
    \item \textbf{Segmentação}: Identificação de diferentes elementos na nuvem de pontos, como paredes, obstáculos e espaços livres.
    \item \textbf{Extração de Características}: Detecção de elementos relevantes como:
          \begin{itemize}
              \item Bordas da pista
              \item Obstáculos dinâmicos
              \item Pontos de referência para localização
          \end{itemize}
\end{itemize}

\subsubsection{Localização e Mapeamento}

A localização precisa é crucial para navegação em alta velocidade.
\cite{Karaman2011Optimal} destacam que o sistema de localização deve integrar
múltiplas fontes de informação:

\begin{itemize}
    \item \textbf{Localização Baseada em Mapa}:
          \begin{equation}
              p(x_t | z_{1:t}, u_{1:t}, m) = \eta p(z_t | x_t, m) \int p(x_t | x_{t-1}, u_t) p(x_{t-1} | z_{1:t-1}, u_{1:t-1}, m) dx_{t-1}
          \end{equation}

    \item \textbf{Mapeamento Online}:
          \begin{equation}
              p(m | z_{1:t}, x_{1:t}) = \prod_i p(m_i | z_{1:t}, x_{1:t})
          \end{equation}
\end{itemize}

\subsubsection{Detecção e Rastreamento de Obstáculos}

Em ambientes de corrida, a detecção e rastreamento de outros veículos é
fundamental. \cite{Ivanov2020Verifying} descrevem que este processo envolve:

\begin{itemize}
    \item \textbf{Detecção}: Identificação de objetos dinâmicos através de:
          \begin{itemize}
              \item Segmentação de nuvem de pontos
              \item Classificação de objetos
              \item Estimação de posição e velocidade
          \end{itemize}

    \item \textbf{Rastreamento}: Acompanhamento de objetos ao longo do tempo usando filtros bayesianos:
          \begin{equation}
              p(x_k | z_{1:k}) = \frac{p(z_k | x_k) p(x_k | z_{1:k-1})}{p(z_k | z_{1:k-1})}
          \end{equation}
\end{itemize}

A integração eficiente destes sistemas de sensoriamento e percepção é crucial
para o desempenho global do veículo autônomo, permitindo uma navegação segura e
eficiente em ambientes dinâmicos de alta velocidade.

\subsection{Algoritmos de Planejamento de Trajetória}

O planejamento de trajetória em veículos autônomos de corrida representa um
desafio complexo que requer a consideração simultânea de múltiplos objetivos:
minimização do tempo de volta, garantia de segurança, respeito às restrições
dinâmicas do veículo e adaptação a condições variáveis da pista. Esta seção
explora os principais algoritmos e técnicas utilizados para enfrentar estes
desafios.

\subsubsection{Planejamento Global de Trajetória}

O planejamento global de trajetória estabelece um caminho ótimo considerando o
conhecimento completo da pista. Segundo \cite{Karaman2011Optimal}, este
processo pode ser formulado como um problema de otimização:

\begin{equation}
    \min_{x(t), u(t)} \int_0^T L(x(t), u(t))dt
\end{equation}

Sujeito às restrições:
\begin{align}
    \dot{x}(t)    & = f(x(t), u(t)) \\
    g(x(t), u(t)) & \leq 0
\end{align}

Onde $x(t)$ representa o estado do veículo, $u(t)$ os comandos de controle,
$L(\cdot)$ a função objetivo (tipicamente tempo ou energia), $f(\cdot)$ a
dinâmica do veículo, e $g(\cdot)$ as restrições do sistema.

\cite{Wang2020LMPC} destacam três abordagens principais para resolver este problema:

\begin{itemize}
    \item \textbf{Métodos Baseados em Amostragem (RRT*, PRM*)}: Estes algoritmos exploram o espaço de estados através de amostragem aleatória, construindo uma árvore ou grafo de trajetórias possíveis. O RRT* (Rapidly-exploring Random Tree*) garante otimalidade assintótica através de reconexão contínua da árvore:
          \begin{equation}
              d_{new} = \min_{x_{near}} \{cost(x_{init} \rightarrow x_{near}) + d(x_{near}, x_{new})\}
          \end{equation}

    \item \textbf{Métodos de Otimização Numérica}: Utilizam técnicas como programação quadrática sequencial (SQP) ou métodos de pontos interiores para encontrar trajetórias ótimas. A função objetivo típica inclui múltiplos termos:
          \begin{equation}
              J = w_t T + w_c \int_0^T \kappa(s)^2 ds + w_s \int_0^T \dot{s}(t)^2 dt
          \end{equation}
          Onde $T$ é o tempo total, $\kappa(s)$ é a curvatura da trajetória, e $\dot{s}(t)$ é a velocidade ao longo do caminho.

    \item \textbf{Métodos Baseados em Curvas Paramétricas}: Utilizam representações matemáticas como splines ou polinômios de Bernstein para gerar trajetórias suaves. Por exemplo, uma spline cúbica pode ser expressa como:
          \begin{equation}
              x(t) = a_0 + a_1t + a_2t^2 + a_3t^3
          \end{equation}
\end{itemize}

\subsubsection{Planejamento Local de Trajetória}

O planejamento local adapta a trajetória global em tempo real para lidar com
obstáculos dinâmicos e condições variáveis. \cite{Stachowicz2023FastRLAP}
apresentam três técnicas principais:

\begin{itemize}
    \item \textbf{Dynamic Window Approach (DWA)}: Considera um espaço de velocidades admissíveis $(v, \omega)$ baseado nas limitações dinâmicas do veículo:
          \begin{equation}
              V_a = \{(v,\omega) | v \in [v_{min}, v_{max}], \omega \in [\omega_{min}, \omega_{max}], \ddot{v} \leq a_{max}, \dot{\omega} \leq \alpha_{max}\}
          \end{equation}

    \item \textbf{Timed Elastic Band (TEB)}: Otimiza uma sequência de poses intermediárias considerando tempo e suavidade:
          \begin{equation}
              F_{total} = \sum_{i} (w_k F_{kin} + w_d F_{dyn} + w_t F_{time} + w_o F_{obs})
          \end{equation}

    \item \textbf{Model Predictive Contouring Control (MPCC)}: Combina seguimento de trajetória com otimização de tempo através de um horizonte deslizante:
          \begin{equation}
              \min_{u(\cdot)} \sum_{k=0}^{N-1} (l_{lag,k}^2 + l_{con,k}^2 - \rho v_{k})
          \end{equation}
          Outras abordagens, como as comparadas por \cite{Misra2024Report}, utilizam RRT* ou campos potenciais gaussianos para o desvio de obstáculos local.
\end{itemize}

\subsubsection{Otimização de Linha de Corrida}

\cite{OKelly2020F1TENTH} explicam que a otimização da linha de corrida envolve encontrar o caminho que minimiza o tempo de volta considerando as características dinâmicas do veículo. Este processo considera:

\begin{itemize}
    \item \textbf{Maximização da Velocidade em Curvas}: A velocidade máxima em uma curva é limitada pela força lateral máxima:
          \begin{equation}
              v_{max} = \sqrt{\mu g R}
          \end{equation}

    \item \textbf{Otimização de Pontos de Entrada e Saída}: A posição lateral na pista é otimizada para maximizar a velocidade média:
          \begin{equation}
              y(s) = \sum_{i=0}^N a_i \phi_i(s)
          \end{equation}
          Onde $\phi_i(s)$ são funções base e $a_i$ são coeficientes a serem otimizados.
\end{itemize}

\subsubsection{Considerações de Segurança e Robustez}

\cite{Ivanov2020Verifying} enfatizam a importância de garantir segurança e robustez no planejamento de trajetória através de:

\begin{itemize}
    \item \textbf{Margens de Segurança Adaptativas}:
          \begin{equation}
              d_{safe}(v) = d_{min} + k_v v + k_a a
          \end{equation}

    \item \textbf{Verificação de Colisão Contínua}:
          \begin{equation}
              d_{obs}(t) > d_{safe}(v(t)) \quad \forall t \in [0,T]
          \end{equation}

    \item \textbf{Consideração de Incertezas}:
          \begin{equation}
              P(collision) \leq \epsilon
          \end{equation}
          Técnicas específicas como o teste de estresse baseado em RRT podem ser usadas para encontrar cenários de falha em manobras complexas como ultrapassagens \cite{Bak2021Stress}.
\end{itemize}

A integração efetiva destes diferentes aspectos do planejamento de trajetória é
crucial para o desenvolvimento de um sistema de corrida autônomo capaz de
operar de forma segura e eficiente em condições de alta velocidade.

\section{Considerações Finais}

\lipsum[23]
