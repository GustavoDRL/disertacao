\chapter{Regulamentações e Dinâmica da Competição F1TENTH}

A categoria F1TENTH estabelece um framework competitivo rigoroso através de
regulamentações técnicas específicas que moldam diretamente o desenvolvimento
dos veículos e as estratégias de corrida~\cite{F1TENTHRules2023}.
Diferentemente de outras competições de robótica, a F1TENTH enfatiza o
desenvolvimento algorítmico em detrimento de vantagens puramente mecânicas,
criando um ambiente que aproxima os desafios enfrentados por veículos autônomos
em escala real~\cite{OBrienF1TENTH2020}.

\section{Especificações Técnicas Obrigatórias e suas Implicações}

\subsection{Plataforma mecânica e propulsão}

As regulamentações dividem a competição em duas categorias distintas: Classe
Restrita e Classe Aberta. Na Classe Restrita, foco principal da competição, o
chassi deve obrigatoriamente ser em escala 1:10, preferencialmente baseado em
modelos Traxxas, permitindo tanto configurações 4WD quanto 2WD
~\cite{F1TENTHRules2023}. Esta padronização não é arbitrária -- visa garantir
que todas as equipes enfrentem desafios físicos comparáveis, evitando vantagens
derivadas exclusivamente de plataformas superiores.

O sistema de propulsão também é rigidamente controlado, limitando-se a motores
DC sem escovas com especificações máximas equivalentes ao Velineon 3500 kV. A
escolha do motor EZRUN 3665SD-3200KV-G3 para o projeto representa uma
implementação adequada, pois seu rating de 3200KV está em conformidade com as
especificações máximas permitidas (3500kV). Além disso, suas dimensões e padrão
de montagem são compatíveis com chassis em escala 1:10, garantindo conformidade
com o regulamento~\cite{F1TENTHRules2023}. Esta restrição impõe um teto de
desempenho que:

\begin{itemize}
      \item Força as equipes a otimizarem seus algoritmos de controle para extrair o máximo
            da plataforma
      \item Minimiza as vantagens derivadas puramente de potência mecânica superior
      \item Cria condições para avaliar a eficiência dos sistemas de navegação e controle
\end{itemize}

\subsection{Sistemas de sensoriamento e computação}

O sistema de sensoriamento incorpora limitações estratégicas que simulam
restrições de orçamento e espaço em veículos reais. O LiDAR principal deve ser
equivalente ou inferior ao Hokuyo UTM-30LX, com alcance máximo de 30 metros,
frequência de varredura de 40 Hz e resolução angular de 0,25 graus
~\cite{GopalarathnamF1TENTH2022}. O LiDAR YDLIDAR X4PRO selecionado para o
projeto atende adequadamente aos requisitos regulamentares, apresentando
especificações compatíveis com os limites estabelecidos: alcance de 0,12--10m
(inferior ao limite de 30m), frequência de varredura ajustável de 6--12Hz e
resolução angular entre 0,43° e 0,86° dependendo da frequência utilizada. Sua
conformidade com a classificação Class I de segurança laser também é um aspecto
importante para atender às normas de segurança da competição
~\cite{F1TENTHRules2023}. Estas especificações restringem a quantidade e
qualidade dos dados disponíveis para percepção ambiental, exigindo:

\begin{itemize}
      \item Algoritmos eficientes de processamento de nuvem de pontos
      \item Técnicas robustas de filtragem para lidar com ruído e reflexões no ambiente de
            competição
      \item Estratégias de fusão sensorial para compensar as limitações do LiDAR
\end{itemize}

Quanto ao sistema computacional, as regulamentações permitem flexibilidade,
exigindo apenas que o processamento seja totalmente realizado a bordo e que o
hardware se adeque fisicamente ao veículo~\cite{F1TENTHRules2023}. A escolha da
Raspberry Pi 4B como plataforma de processamento satisfaz plenamente estes
requisitos, oferecendo processamento suficiente para algoritmos de navegação e
controle em um formato compacto que se adequa facilmente ao chassi do veículo.
Sua ampla compatibilidade com frameworks de software como ROS2 facilita a
implementação dos algoritmos necessários para a competição. Esta flexibilidade
reconhece a importância do poder computacional para algoritmos avançados de
percepção e controle, permitindo o uso de plataformas como Nvidia Jetson, Intel
NUC ou Raspberry Pi, conforme as necessidades específicas de processamento da
solução implementada.

O controlador eletrônico Makerbase VESC MINI 6.7 representa uma escolha
estratégica para o projeto, visto que suas capacidades avançadas de controle de
motor BLDC e PMSM, incluindo recursos de posicionamento preciso e controle de
velocidade/corrente, permitem a implementação de algoritmos sofisticados de
controle. Suas interfaces de comunicação múltiplas (USB, CAN, UART) facilitam a
integração com o sistema ROS2 executado na Raspberry Pi, criando um sistema de
controle robusto e responsivo que atende aos requisitos de processamento a
bordo exigidos pelo regulamento~\cite{F1TENTHRules2023}.

\subsection{Requisitos de segurança obrigatórios}

Os requisitos de segurança não são meras formalidades, mas elementos cruciais
que simulam preocupações de veículos autônomos em escala real. Todo veículo
deve incorporar:

\begin{itemize}
      \item Para-choques frontais e traseiros funcionais
      \item Área mínima de detecção por LiDAR de 12×12 centímetros entre 10--30 centímetros do solo
      \item Sistema de parada de emergência acionável remotamente
      \item Algoritmos de detecção e prevenção de colisões demonstráveis
\end{itemize}

A configuração do sistema VESC MINI 6.7 com capacidade de frenagem regenerativa
e limites de corrente programáveis contribui significativamente para atender
aos requisitos de segurança da competição, permitindo implementar um sistema
eficaz de parada de emergência. Sua capacidade de resposta rápida (com
limitação de corrente e proteções integradas) permite implementar algoritmos
confiáveis de detecção e prevenção de colisões, conforme exigido pelo
regulamento~\cite{MoralesF1TENTH2021}.

Estes requisitos são testados rigorosamente antes da competição, e falhas em
qualquer um deles resultam em desqualificação~\cite{MoralesF1TENTH2021}.
Particularmente para as corridas head-to-head, os veículos devem passar por
testes de evitação de obstáculos estáticos e dinâmicos, simulando situações
reais de tráfego.

\section{Ambientes de Competição e suas Características Técnicas}

O ambiente de corrida é meticulosamente projetado para apresentar desafios
específicos aos sistemas autônomos. A pista, construída em superfície plana e
reflexiva com bordas formadas por dutos de ar de 33 cm de diâmetro, cria
condições particularmente desafiadoras para sistemas de percepção
~\cite{OKelly2020F1TENTH}:

\begin{itemize}
      \item Os feixes do LiDAR frequentemente refletem no solo reflexivo, gerando ruídos e
            falsas leituras
      \item As câmeras de profundidade enfrentam dificuldades com a detecção adequada dos
            limites da pista
      \item As bordas curvas dos dutos de ar criam padrões de reflexão não-lineares e
            complexos
\end{itemize}

O LiDAR YDLIDAR X4PRO apresenta características que podem ajudar a lidar com
essas condições desafiadoras da pista, incluindo sua forte resistência a
interferências de luz ambiente (tolerância de até 40000 Lux) e capacidade de
medição estável em superfícies com diferentes refletividades. A resolução
angular configurável permite ajustes para obter o equilíbrio ideal entre
densidade de pontos e frequência de atualizações, fundamentais para a navegação
precisa nas condições específicas da competição~\cite{OKelly2020F1TENTH}.

A largura mínima de 90 centímetros (equivalente a três larguras de carro) é
especificamente calculada para permitir manobras de ultrapassagem durante as
corridas head-to-head, exigindo sistemas sofisticados de planejamento de
trajetória. O circuito combina intencionalmente elementos diversos como:

\begin{itemize}
      \item Pontos de estrangulamento que forçam decisões estratégicas de posicionamento
      \item Curvas fechadas que testam os limites dos algoritmos de controle
      \item Retas que permitem maximização de velocidade e avaliação de sistemas de
            controle longitudinal
      \item Curvas extra-largas que abrem possibilidades para múltiplas linhas de corrida
            otimizadas
\end{itemize}

A combinação do motor EZRUN 3665SD-3200KV com o controlador VESC MINI 6.7
oferece uma resposta dinâmica adequada para enfrentar estes diversos elementos
da pista. O motor fornece torque suficiente para aceleração rápida nas retas,
enquanto o controle preciso oferecido pelo VESC permite ajustes finos nas
curvas e manobras de ultrapassagem. A capacidade de programação avançada do
VESC possibilita a implementação de diferentes perfis de controle otimizados
para cada seção do circuito~\cite{BalachandranF1TENTH2020}.

Esta diversidade de elementos não é aleatória, mas cuidadosamente planejada
para testar a versatilidade dos algoritmos implementados sob diferentes
condições, simulando os variados cenários que veículos autônomos reais
enfrentam~\cite{BalachandranF1TENTH2020}.

\section{Estrutura e Funcionamento da Competição}

\subsection{Inspeção técnica e verificação de conformidade}

A competição inicia com uma rigorosa fase de inspeção técnica, onde cada
veículo é minuciosamente avaliado quanto à conformidade com as especificações
técnicas~\cite{F1TENTHRules2023}. Esta etapa inclui:

\begin{itemize}
      \item Verificação dimensional do chassi e componentes
      \item Teste dos sistemas de parada de emergência sob diferentes condições
      \item Validação das especificações do LiDAR e outros sensores
      \item Confirmação de que todo processamento ocorre a bordo
\end{itemize}

Os componentes selecionados para o projeto – motor EZRUN 3665SD-3200KV, VESC
MINI 6.7, LiDAR YDLIDAR X4PRO e Raspberry Pi 4B – foram escolhidos considerando
sua conformidade com os requisitos da inspeção técnica. As especificações
detalhadas de cada componente facilitam a documentação necessária para este
processo, enquanto as capacidades de diagnóstico do sistema VESC e a interface
de software ROS2 permitem demonstrar claramente o funcionamento dos sistemas
críticos durante a avaliação~\cite{F1TENTHRules2023}.

Veículos que não atendem integralmente às especificações são desclassificados
ou reclassificados para a Classe Aberta, dependendo da natureza da
não-conformidade.

\subsection{Time Trial: avaliação de desempenho individual}

O Time Trial constitui a segunda fase da competição, estruturada em duas
baterias de 5 minutos cada. Nesta modalidade, cada veículo compete
individualmente na pista, buscando:

\begin{itemize}
      \item Completar o maior número possível de voltas consecutivas sem interrupções
      \item Registrar a volta mais rápida possível durante o período alocado
\end{itemize}

O sistema de pontuação é deliberadamente projetado para equilibrar velocidade e
consistência, combinando o tempo da volta mais rápida com o número de voltas
consecutivas completadas sem intervenção humana~\cite{ShinF1TENTH2020}. Esta
abordagem força as equipes a desenvolverem soluções que balanceiem:

\begin{itemize}
      \item Algoritmos agressivos de otimização de trajetória para velocidade máxima
      \item Sistemas robustos de controle que garantam consistência entre múltiplas voltas
      \item Estratégias de gerenciamento de risco que minimizem a probabilidade de falhas
\end{itemize}

A integração do VESC MINI 6.7 com ROS2 conforme detalhado na documentação do
projeto proporciona um sistema de controle preciso e adaptável, crucial para o
desempenho consistente na fase de Time Trial. A capacidade de ajustar
parâmetros de controle PID e perfis de velocidade permite otimizar o veículo
tanto para voltas rápidas quanto para desempenho consistente ao longo de
múltiplas voltas, alinhando-se perfeitamente com os critérios de pontuação
desta fase~\cite{ShinF1TENTH2020}.

\subsection{Corridas Head-to-Head: o desafio máximo da autonomia}

A fase final e mais complexa é a Corrida Head-to-Head, onde dois veículos
competem simultaneamente na pista. Esta modalidade representa o ápice dos
desafios de autonomia, exigindo:

\begin{itemize}
      \item Sistemas robustos de detecção e rastreamento de outros veículos em movimento
      \item Algoritmos adaptativos de planejamento de trajetória que considerem obstáculos
            dinâmicos
      \item Estratégias sofisticadas de ultrapassagem que calculem riscos e oportunidades
      \item Capacidade de ajuste dinâmico da linha de corrida baseado na posição do
            oponente
\end{itemize}

O LiDAR YDLIDAR X4PRO, com sua capacidade de varredura de 360° e alta
frequência de amostragem (5kHz), fornece dados cruciais para a detecção de
outros veículos durante as corridas head-to-head. A resolução angular permite
identificar com precisão a posição do oponente, enquanto a Raspberry Pi 4B
oferece o poder de processamento necessário para executar algoritmos complexos
de planejamento de trajetória e tomada de decisões em tempo real, fundamentais
para implementar estratégias eficazes de ultrapassagem
~\cite{GopalarathnamF1TENTH2022}.

As corridas são organizadas em formato de eliminatórias, com os veículos mais
rápidos das fases de Time Trial recebendo posições favoráveis no grid
~\cite{GopalarathnamF1TENTH2022}. Uma característica crucial desta fase é a
necessidade de combinar velocidade máxima com tomada de decisão defensiva e
ofensiva, similar aos desafios enfrentados por veículos autônomos em ambientes
urbanos densos.

\section{Implicações das Regulamentações para o Desenvolvimento Customizado}

As regulamentações da F1TENTH impactam diretamente as decisões de projeto no
desenvolvimento de veículos customizados. Considerando o desenvolvimento de uma
versão personalizada que atenda às regulações, é necessário:

\begin{enumerate}
      \item Equilibrar o poder computacional com restrições de tamanho e energia, optando
            por plataformas que ofereçam desempenho otimizado para algoritmos específicos
            implementados. A Raspberry Pi 4B representa este equilíbrio, oferecendo
            processamento suficiente em um formato compacto e com consumo energético
            moderado compatível com as baterias típicas de veículos RC em escala 1:10.

      \item Desenvolver estratégias de fusão sensorial que compensem as limitações do LiDAR
            principal, potencialmente incorporando dados de IMUs, encoders e câmeras dentro
            dos limites regulatórios~\cite{YangF1TENTH2022}. A integração entre o LiDAR
            YDLIDAR X4PRO e os encoders disponíveis através do VESC MINI 6.7 possibilita
            implementar fusão sensorial eficaz para melhorar a percepção e navegação.

      \item Implementar sistemas de controle adaptáveis que possam alternar entre modos de
            operação otimizados para diferentes segmentos da pista e situações
            competitivas. O VESC MINI 6.7, com sua capacidade de programação avançada e
            ajuste de parâmetros em tempo real via comunicação com a Raspberry Pi, permite
            a implementação destes sistemas de controle adaptáveis.

      \item Criar algoritmos de planejamento que incorporem tanto conhecimento a priori do
            circuito quanto capacidade de adaptação a condições dinâmicas durante as
            corridas head-to-head~\cite{MoralesF1TENTH2021}

      \item Projetar sistemas de segurança que não apenas atendam aos requisitos mínimos,
            mas que também sejam suficientemente robustos para permitir operação agressiva
            próxima aos limites físicos do veículo. As proteções integradas do VESC MINI
            6.7 (sobrecorrente, sobretensão, temperatura) fornecem uma base sólida para
            implementar estes sistemas de segurança.
\end{enumerate}

As regulamentações, embora restritivas, criam um ambiente de inovação focada
onde soluções algorítmicas sofisticadas ganham precedência sobre vantagens de
hardware, refletindo adequadamente os desafios enfrentados no desenvolvimento
de veículos autônomos em escala real~\cite{OKelly2020F1TENTH}. Os componentes
selecionados para este projeto demonstram um equilíbrio adequado entre
desempenho e conformidade regulatória, proporcionando uma plataforma viável
para competir efetivamente na categoria F1TENTH.